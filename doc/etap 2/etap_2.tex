\documentclass{../llncs_template/llncs} 
\pagestyle{plain}
\usepackage[utf8]{inputenc}
\usepackage{polski}
\usepackage[polish]{babel}
\usepackage{makeidx}
\usepackage{algorithm}
\usepackage{algpseudocode}
%\usepackage{hyperref} % powoduje buga w LLNCS i nie możemy go użyć :(
\begin{document}

\title{Teoria algorytmów i obliczeń -- Projekt -- Etap 2}
\author{Błażej Bobko, Jakub Gocławski, Patryk Kujawski, Radosław Kutkowski}
\institute{Wydział Matematyki i Nauk Informacyjnych, Politechnika Warszawska}
\maketitle 

\section{Szczegóły implementacyjne}

Rozwiązanie zostało zaimplementowane w języku $C\#$ z wykorzystaniem technologii \emph{WPF} do stworzenia środowiska graficznego (\emph{GUI}). Nie były wykorzystywane żadne dodatkowe, zewnętrzne biblioteki.

\subsection{Moduły}

Projekt został podzielony na 4 moduły opisane poniżej.

\subsubsection*{UserInterface}
Moduł zawiera implementację \emph{GUI} w technologii \emph{WPF} oraz klasy pomocnicze wspomagające interakcję z użytkownikiem, m.in. w celu wczytania automatu z pliku lub wprowadzenia parametrów obliczeń.

\subsubsection*{TestGenerator}
Moduł zawiera klasę \emph{TestSets}, służącą do przechowywania danych zbiorów: treningowego oraz testowego. Klasa ta potrafi także generować te zbiory, a także zapisywać je do pliku i wczytywać z pliku. Na podstawie danych zawartych w tej klasie można odpowiedzieć na pytanie, czy dwa słowa są w relacji.

\subsubsection*{PSO}
Moduł zawiera klasę \emph{MachinePSO} zawierającą główną pętlę algorytmu \emph{PSO} oraz klasę \emph{Particle} będącą reprezentacją pojedynczej cząsteczki roju wykorzystywanego przez \emph{MachinePSO}.

\subsubsection*{LanguageProcessor}
Moduł zawiera klasę \emph{Machine} będącą reprezentacją odtwarzanego automatu, klasę \emph{Alphabet} zawierającą dostępne litery alfabetu wraz z funkcjami ułatwiającymi konwersję ich formatu oraz klasę pomocniczą \emph{Extensions}.

\section{Interfejs użytkownika}
Po uruchomieniu programu możliwa jest modyfikacja głównych parametrów obliczeń. Oprócz podstawowych parametrów dokładniej opisanych w dokumencie z I etapu obecne są także dwa dodatkowe:

\begin{itemize}
\item Szansa śmierci cząsteczki -- cząsteczka z podanym prawdopodobieństwem umrze i zostanie zastąpiona nową, losową cząsteczką
\item Cząsteczki przekazane do następnej iteracji -- liczba cząsteczek z $n$-tej iteracji, które zostaną przekazane do $n+1$-szej iteracji, gdzie $n$, to liczba stanów poszukiwanego automatu
\end{itemize}

Parametry należy zatwierdzić przyciskiem \emph{Zatwierdź Parametry}. Następnie należy kliknąć przycisk \emph{Wczytaj automat}. Pokaże się okno wyboru pliku, w którym należy wskazać plik w zdefiniowanym w zadaniu formacie. Po wczytaniu pliku należy kliknąć przycisk \emph{Generuj zbiory}.

Pojawi się nowe okno umożliwiające albo wczytanie wygenerowanych wcześniej testów z pliku albo wygenerowanie nowych zbiorów testowych i treningowych.

\newpage

 Podczas generowania zbiorów testowych można ustalić parametry:

\begin{itemize}
\item Losowe Testy -- liczba losowych słów dłuższych niż parametr \emph{Dokładne próby}
\item Dokładne próby -- zostaną wygenerowane wszystkie słowa o długości $<=$ od podanego parametru
\item Grupa Kontrolna -- rozmiar zbioru par dłuższych słów, nie bioracych udziału w procesie optymalizacji w trakcie trwanis PSO
\end{itemize}

Po wygenerowaniu zbiorów lub wczytaniu ich z pliku, należy nacisnąć przycisk \emph{Akceptuj}. Po powrocie do głównego okna programu, aktywny staje się przycisk \emph{Rozpocznij PSO}; jego kliknięcie powoduje pojawienie się nowego okna, w którym można rozpocząć obliczenia. Aby to zrobić, należy kliknąć przycisk \emph{Start}. W trakcie trwania obliczeń można śledzić ich postęp. Po zakończeniu działania algorytmu pojawi się okno z komunikatem. Następnie można zapisać logi z obliczeń za pomocą przycisku \emph{Stwórz log}.

\section{Dodatkowe funkcjonalności}

\subsection{Interfejs graficzny}
Co prawda element ten nie był jeszcze wymagany na tym etapie, jednak został już zaimplementowany, co ułatwia uruchamianie obliczeń. Interfejs jest w pełni responsywny w trakcie trwania obliczeń i pozwala na bieżąco śledzić postępy w poszukiwaniu rozwiązania.

\subsection{Zapisywanie i wczytywanie zbiorów testowych}
Raz wygenerowane pozwalają znacznie przyspieszyć porównywanie wydajności programu dla różnych parametrów oraz zapewniają, że próby odtwarzania będą dotyczyć dokładnie tego samego automatu.

\subsection{Równoległe wykonywanie obliczeń}
Główna pętla programu została zaimplementowana z wykorzystaniem wątków za pomocą dostępnej w $C\#$ klasy $Task$.

\subsection{Zapisywanie logów}
Po wykonaniu obliczeń możliwe jest zapisanie do pliku pełnego logu zawierającego wyniki dla każdej testowanej pary słów.

\section{Zmiany względem I etapu}

\subsection{Reprezentacja automatu w trakcie działania PSO}

W części opisującej działanie algorytmu, w dokumencie z poprzedniego etapu, planowaliśmy reprezentować automat jako macierz zmiennopozycyjnych liczb. Macierz ta miała mieć zaokrąglane wartości do liczb całkowitych i na jej podstawie miał być konstruowany automat, za pomocą którego obliczana byłaby wartość funkcji błędu.

Zdecydowaliśmy się jednak wprowadzić usprawnienie polegające na tym, że cząsteczka posiada automat reprezentowany przez macierz z liczbami zmiennopozycyjnymi, a są one zaokrąglanie do liczb całkowitych tylko w momencie przeprowadzania obliczenia na automacie. Dzięki temu uniknęliśmy kosztownego tworzenia nowych obiektów automatów, a w szczególności kopiowania macierzy z wartościami. Teraz obliczenia odbywają się ,,w miejscu'' z perspektywy pamięci.

\end{document}
