\documentclass{../llncs_template/llncs} 
\pagestyle{plain}
\usepackage[utf8]{inputenc}
\usepackage{polski}
\usepackage[polish]{babel}
\usepackage{makeidx}
\usepackage{algorithm}
\usepackage{algpseudocode}
%\usepackage{hyperref} % powoduje buga w LLNCS i nie możemy go użyć :(
\begin{document}

\title{Teoria algorytmów i obliczeń -- Projekt -- Etap 2}
\author{Błażej Bobko, Jakub Gocławski, Patryk Kujawski, Radosław Kutkowski}
\institute{Wydział Matematyki i Nauk Informacyjnych, Politechnika Warszawska}
\maketitle 

\section{Szczegóły implementacyjne}

Rozwiązanie zostało zaimplementowane w języku $C\#$ z wykorzystaniem technologii \emph{WPF} do stworzenia środowiska graficznego (\emph{GUI}). Nie były wykorzystywane żadne dodatkowe, zewnętrzne biblioteki.

\subsection{Moduły}

Projekt został podzielony na 4 moduły opisane poniżej.

\subsubsection*{UserInterface}
Moduł zawiera implementację \emph{GUI} w technologii \emph{WPF} oraz klasy pomocnicze wspomagające interakcję z użytkownikiem, m.in. w celu wczytania automatu z pliku lub wprowadzenia parametrów obliczeń.

\subsubsection*{TestGenerator}
Moduł zawiera klasę \emph{TestSets}, służącą do przechowywania danych zbiorów: treningowego oraz testowego. Klasa ta potrafi także generować te zbiory, a także zapisywać je do pliku i wczytywać z pliku. Na podstawie danych zawartych w tej klasie można odpowiedzieć na pytanie, czy dwa słowa są w relacji.

\subsubsection*{PSO}
Moduł zawiera klasę \emph{MachinePSO} zawierającą główną pętlę algorytmu \emph{PSO} oraz klasę \emph{Particle} będącą reprezentacją pojedynczej cząsteczki roju wykorzystywanego przez \emph{MachinePSO}.

\subsubsection*{LanguageProcessor}
Moduł zawiera klasę \emph{Machine} będącą reprezentacją odtwarzanego automatu, klasę \emph{Alphabet} zawierającą dostępne litery alfabetu wraz z funkcjami ułatwiającymi konwersję ich formatu oraz klasę pomocniczą \emph{Extensions}.


\section{Dodatkowe funkcjonalności}

\subsection{Interfejs graficzny}
Co prawda element ten nie był jeszcze wymagany na tym etapie, jednak został już zaimplementowany, co ułatwia uruchamianie obliczeń. Interfejs jest w pełni responsywny w trakcie trwania obliczeń i pozwala na bieżąco śledzić postępy w poszukiwaniu rozwiązania.

\subsection{Zapisywanie i wczytywanie zbiorów testowych}
Raz wygenerowane pozwalają znacznie przyspieszyć porównywanie wydajności programu dla różnych parametrów oraz zapewniają, że próby odtwarzania będą dotyczyć dokładnie tego samego automatu.

\subsection{Równoległe wykonywanie obliczeń}

\subsection{Zapisywanie logów}

\section{Zmiany względem I etapu}

\subsection{Reprezentacja automatu w trakcie działania PSO}

\end{document}
