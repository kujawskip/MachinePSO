\documentclass{../llncs_template_final/llncs} 
\pagestyle{plain}
\usepackage[utf8]{inputenc}
\usepackage{polski}
\usepackage[polish]{babel}
\usepackage{makeidx}
\usepackage{algorithm}
\usepackage{algpseudocode}
%\usepackage{hyperref} % powoduje buga w LLNCS i nie możemy go użyć :(
\begin{document}

\title{Teoria algorytmów i obliczeń -- Projekt -- Etap 3}
\author{Błażej Bobko, Jakub Gocławski, Patryk Kujawski, Radosław Kutkowski}
\institute{Wydział Matematyki i Nauk Informacyjnych, Politechnika Warszawska}
\maketitle 

\section{Dokumentacja algorytmu}

\section{Dokumentacja techniczna rozwiązania}

\section{Raport z testów}

\subsection{Parametry obliczeń}

\begin{itemize}
\item Liczba liter w alfabecie: 5
\item Liczba cząsteczek: 20
\item Waga prędkości (inertia weight): 0,729
\item Waga lokalna: (cognitive weight): 1,49445
\item Waga globalna: (social weight): 1,49445	
\item Szansa śmierci cząsteczki: 0,01
\item Cząsteczki przekazane do następnej iteracji: 4
\item Maksymalna liczba stanów: (różne wartosci w częsci A oraz B)
\end{itemize}

\subsection{Testy dla stałej c=4}
Obliczenia były wkonywane dla stałej $c = 5$. A zatem:
\begin{itemize}
\item Liczba wszystkich słów krótkich: 780
\item Liczba wszystkich permutacji słów krótkich: 303 810
\item A zatem rozmiar zbioru treningowego: 607 620
\item Rozmiar zbioru testowego: 607 620
\end{itemize}


\subsection{Testy dla stałej c=5}
Nie udało się wykonać obliczeń dla stałej $c = 5$, gdyż taka wartosć ma nastepujące konsekwencje:
\begin{itemize}
\item Liczba wszystkich słów krótkich: 3 905
\item Liczba wszystkich permutacji słów krótkich: 7 622 560
\item A zatem rozmiar zbioru treningowego: 15 245 120
\item Rozmiar zbioru testowego: 15 245 120
\end{itemize}

Powoduje to problemy z pamięcią oraz wydajnoscią. Sam proces generowania zbiorów o takiej licznosci wymaga w szczytowym momencie 10,5 GB pamięci RAM. Wymaga to bardzo mocnego komputera oraz wykonywania w rodowisku 64-bitowym.
Niestety nawet po wygenerowaniu tak ogromnych zbiorów obliczenia trwają bardzo długo. Próbowalismy uruchomić obliczenia dla Automatu nr 2 z podpunktu A, klasa 5-stanowa. Po 25 minutach obliczeń (z wykorzystaniem 11 GB pamięci RAM), przetwarzany był ciągle automat o 2 stanach, z błędem 45\%. 

Przerwalimy obliczenia, nie widząc perspektyw na ich ukończenie, a wolelimy skupić się na dokładniejszych testach dla stałej $c = 4$.

\subsection{A. Rekonstrukcja automatów}

\subsection{B. Aproksymacja automatów}

\end{document}
